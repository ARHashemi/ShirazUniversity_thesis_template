\chapter{توسعه بسته محاسبات عددی بر پایه\lr{FDTD}}\label{chap2}
در این فصل به تفصیل چگونگی ایجاد و توسعه بسته محاسبات عددی به کار رفته در این پژوهش شرح داده می‌شود. تمامی شبیه‌سازی‌های پژوهش حاضر با استفاده از این بسته انجام پذیرفته است. به تناسب نوع استفاده مورد نیاز در این پژوهش، شیوه‌های خاصی برای به کار بردن \lr{FDTD} استفاده شده است که برخی (مانند نحوه اعمال بیم گاوسی) ممکن است با آنچه در سایر مراجع معرفی شده، متفاوت و نو باشد. بنا بر این لازم است ضمن آشنایی مقدماتی با پایه‌های روش عددی، موارد افزوده شده و اصلاحات انجام‌گرفته در این روش نیز به صورت مشروح توضیح داده شوند.

در بخش اول به صورت مختصر اساس روش عددی را بیان می‌کنیم و در ادامه، مواردی مانند منابع موج فرودی و چگونگی افزودن آن‌ها به بسته، تغییر پروفایل عرضی تابش فرودی، افزودن مواد پاشنده با استفاده از مدل درود-لورنتس و ... بیان خواهد شد. برخی مطالب این فصل بر گرفته از مراجع است که در جای خود ذکر شده و بخشی از مطالب و روش‌های بیان شده از موارد بدیعی است که متناسب با نیاز این پژوهش به روش مرسوم اضافه شده است.
